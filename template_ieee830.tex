% Created 2020-04-24 vie 23:03
% Intended LaTeX compiler: pdflatex
\documentclass[12pt,a4paper]{article}
%------------------------------ PACKAGES -----------------------------%
\usepackage[utf8]{inputenc}
\usepackage[T1]{fontenc}
\usepackage{graphicx}
\usepackage{grffile}
\usepackage{longtable}
\usepackage{wrapfig}
\usepackage{rotating}
\usepackage[normalem]{ulem}
\usepackage{amsmath}
\usepackage{textcomp}
\usepackage{amssymb}
\usepackage{capt-of}
\usepackage{hyperref}
\usepackage[table]{xcolor}
\usepackage[left=2.00cm, right=2.50cm, top=5cm, bottom=2.00cm]{geometry}
\usepackage{fancyhdr}
\usepackage{tabularx}

%------------------------------ DEFINES ------------------------------%
\def\Author{Lautaro Vera}
\def\Email{lautarovera93@gmail.com}
\def\Title{Bootloader Multi-Etapa}

%------------------------------ SETTINGS -----------------------------%
\fancyhead[RO,L]{\thepage}
\fancyhead[LO]{\emph{\uppercase{\leftmark}}}
\fancyfoot{}
\setlength\headheight{15pt} 
\renewcommand{\headrulewidth}{1.0pt}
\renewcommand{\contentsname}{Índice}
\pagestyle{fancy}
\hypersetup
{
	colorlinks = true,
	linkcolor = {black},
	linkbordercolor = {white},
	urlcolor = {blue},
	pdfauthor={},
	pdftitle={ERS-IEEE-830},
	pdfkeywords={},
	pdfsubject={},
	pdfcreator={Emacs 26.2 (Org mode 9.1.9)}, 
	pdflang={Spanish}
}

%----------------------------- DOCUMENT -----------------------------%
\begin{document}
	\begin{titlepage}
		\centering
		{\scshape\Huge \Title \par}
		\vspace{5cm}
		{\bfseries\LARGE Especificación de Requerimientos de Software \par}
		\vspace{1cm}
		{\scshape\Large IEEE Std. 830-1998 \par}
		\vfill
		{\Large \Author \\ (\href{mailto:\Email}{\Email})\par}
		\vfill
		{7 de Noviembre 2021 \par}
	\end{titlepage}

\newgeometry{left=2.00cm, right=2.50cm, top=2.50cm, bottom=2.00cm}

\section*{Registro de cambios}
\label{sec:changelog}


\begin{table}[ht]
\label{tab:changelog}
	\centering
	\begin{tabularx}{\linewidth}{@{}|c|X|c|c|@{}}
	\hline
	\rowcolor[HTML]{C0C0C0} 
	Versión & \multicolumn{1}{c|}{\cellcolor[HTML]{C0C0C0}Descripción} & Autor       & Fecha      \\ \hline
	A      & Versión original                                          & \Author & 07/10/2021 \\ \hline
	%1      & Se completa hasta el punto 5 inclusive                 & 04/11/2021 \\ \hline
	%2      & Se completa hasta el punto 7 inclusive
	%		  Se puede agregar algo más \newline
	%		  En distintas líneas \newline
	%		  Así                                                    & dd/mm/aaaa \\ \hline
	%3      & Se completa hasta el punto 11 inclusive                & dd/mm/aaaa \\ \hline
	%4      & Se completa el plan	                                 & dd/mm/aaaa \\ \hline
	\end{tabularx}
\end{table}

\pagebreak

\tableofcontents

\newpage

\section{Introducción}
\label{sec:introduction}


\subsection{Propósito}
\label{sec:purpose}

Este documento define la especificación de requerimientos de software para el bootloader de un dispositivo médico de 
electrocardiografía ambulatoria (Holter).

Está dirigido a clientes y desarrolladores dedicados al análisis, diseño e implementación de sistemas críticos.


\subsection{Ámbito del sistema}
\label{sec:scope}

\begin{itemize}
	\item Será propiedad de ECCOSUR, empresa auspiaciante del proyecto.

	\item Realizará sus funciones con la seguridad e integridad de un sistema crítico.

	\item Se espera lograr una evolución considerable respecto del bootloader actual que posee el Holter de ECCOSUR.
	
	\item Se podrá integrar con otros sistemas embebidos que posean la misma arquitectura del Holter de ECCOSUR.
\end{itemize}


\subsection{Definiciones, Acrónimos y Abreviaturas}
\label{sec:glossary}

\begin{table}[ht]
\label{tab:glossary}
	\centering
	\begin{tabularx}{\linewidth}{@{}|>{\columncolor[gray]{0.8}}l|X|@{}}
	\hline
	AKA    			& \textbf{A}lso \textbf{K}nown \textbf{A}s																			\\ \hline
	ARM     		& \textbf{A}dvanced \textbf{R}ISC \textbf{M}achine	     															\\ \hline	
	Bootmanager     & Programa de arranque primario, también llamado bootloader primario												\\ \hline
	Bootloader      & Programa de arranque secundario, también llamado bootloader secundario											\\ \hline
	Branching     	& Técnica de salto de memoria para pasar de aplicación a bootloader y visceversa									\\ \hline
	Checksum     	& Suma de verificación, i.e. función de redundancia para detectar cambios accidentales en una secuencia de datos 	\\ \hline
	Holter     		& Electrocardiógrafo ambulatorio     																				\\ \hline
	IEEE     		& \textbf{I}nstitute of \textbf{E}lectrical and \textbf{E}lectronics \textit{E}ngineers								\\ \hline
	IDE     		& \textbf{I}ntegrated \textbf{D}evelopment \textbf{E}nvironment														\\ \hline
	LWPAN     		& \textbf{L}ow \textbf{P}ower \textbf{W}ide \textit{A}rea \textit{N}etwork											\\ \hline
	MCU     		& \textbf{M}icro-\textbf{C}ontroller \textbf{U}nit																	\\ \hline
	NDA     		& \textbf{N}on-\textbf{D}isclosure \textbf{A}greement																\\ \hline
	NIR     		& \textbf{N}on-\textbf{I}nitialized \textbf{R}AM	
								\\ \hline
	Parsing    		& Técnica de análisis de sintáxis de una secuencia con el objetivo de detectar patrones								\\ \hline
	POR     		& \textbf{P}ower \textbf{O}n \textbf{R}eset	
								\\ \hline
	RISC			& \textbf{R}educed \textbf{I}nstruction \textbf{S}et \textbf{C}omputer												\\ \hline
	Secureboot     	& Técnica que impide la ejecución de cualquier software no firmado o certificado									\\ \hline
	Stack	     	& En protocolos de comunicación, la implementación en código del modelo OSI del protocolo							\\ \hline
	Std.     		& Standard, i.e. estándar							     															\\ \hline
	TI       		& \textbf{T}exas \textbf{I}nstruments				     															\\ \hline
	USB     		& \textbf{U}niversal \textbf{S}erial \textbf{B}us, i.e. bus universal en serie										\\ \hline
	\end{tabularx}
\end{table}


\subsection{Referencias}
\label{sec:references}

En esta subsección se mostrará una lista completa de todos los
documentos referenciados en la ERS


\subsection{Visión general del documento}
\label{sec:overview}

Este documento se realiza siguiendo el estándar IEEE Std. 830-1998.


\section{Descripción general}
\label{sec:general_description}


\subsection{Perspectiva del producto}
\label{sec:product_perspective}

\begin{itemize}
	\item El bootloader es un componente de software que se almacena en la memoria de programa de manera independiente. De esta manera tiene un flujo 	de programa propio y puede gestionar el borrado y escritura de la memoria de programa de aplicación. Se integra de manera complementaria con la 		aplicación de software, y la provee de actualización, restauración y validación.
	
	\item Como todo bootloader, el producto tiene una fuerte dependencia con la arquitectura sobre la que está embarcado. 
	No tiene relación con el código de firmware del Holter, ya que trata las imágenes de manera transparente. En otras palabras, el bootloader 
	funcionará de la misma manera en cualquier sistema embebido que posea la arquitectura del Holter, siempre que haya sido correctamente integrado.
	
	\item El bootloader soportará la descarga de imágenes de firmware a través del protocolo USB. Por lo tanto, se corresponde unívocamente 
	con el sistema "host" que soporta dicho protocolo.
\end{itemize}


\subsection{Funciones del producto}
\label{sec:product_functions}

\begin{itemize}
	\item Actualización del firmware actual a versiones nuevas certificadas (\textit{AKA Update}).
	\item Restauración del firmware actual a la última versión previa estable instalada (\textit{AKA Rollback}).
	\item Validación de origen e integridad del firmware a instalar (\textit AKA Secureboot).
	\item Soporte del formato de imágenes binarias \textbf{Motorola S-record} (\textit{AKA SREC Parsing}).
\end{itemize}


\subsection{Características de los usuarios}
\label{sec:user_characteristics}

Los usuarios finales de este producto son personas capacitadas en el ámbito de los sistemas embebidos.

Para poder seguir correctamente los pasos de integración del bootloader el usuario debe poseer conocimientos avanzados de software embebido, lenguaje C y compilador \textit{ARM-CGT de Texas Instruments}.


\subsection{Restricciones}
\label{sec:constraints}

\begin{itemize}
	\item El software del bootloader debe mantenerse bajo control de versiones GIT.
	
	\item La difusión del desarrollo está regulada por un documento NDA firmado por las partes interesadas.

	\item El alcance del bootloader está limitado a la arquitectura \textit{MSP430 de Texas Instruments} sobre la que será desarrollado, y sólo podrá 
	ser integrado con sistemas embebidos que tengan dicha arquitectura.

	\item El protocolo de comunicación para el caso del Holter está limitado a USB ya que es el que está implementado en el hardware existente. Para 
	futuras integraciones, el núcleo del bootloader puede complementarse con distintos \textit{stacks} de comunicación.

	\item El hardware existente no dispone de memoria flash externa, por lo que tanto el bootloader, la aplicación y las imágenes de actualización y 
	restauración deben ser gestionadas en la misma memoria flash interna del microcontrolador.

	\item El tamaño de programa debe ser el menor posible para brindar mayor espacio a la aplicación, por lo que código del software está 
	restringido a lenguaje C.
	
	\item El éstandar de C a utilizar debe ser CXX (TBD).

	\item Se deben tener conocimientos del compilador TI ARM-CGT, en particular de Linker Scripts.
	
	\item La versión del compilador TI ARM-CGT a utilizar debe ser 20.2.2.LTS para compatabilidad con desarrollos existentes.

	\item Toda documentación relativa al código debe ser generada a través de la herramienta Doxygen.
\end{itemize}


\subsection{Suposiciones y dependencias}
\label{sec:assumptions_dependencies}

Esta subsección de la ERS describirá aquellos factores que, si
cambian, pueden afectar a los requisitos. Por ejemplo, los
requisitos pueden presuponer una cierta organización de ciertas
unidades de la empresa, o pueden presuponer que  el sistema correrá
sobre cierto sistema operativo. Si cambian dichos detalles en la
organización de la empresa, o si cambian ciertos detalles técnicos,
como el sistema operativo, puede ser necesario revisar y cambiar los
requisitos.

\begin{itemize}
	\item Se asume el acceso total al código fuente de la aplicación y bootloader actual del dispositivo Holter.
	
	\item Se asume la independencia en el uso de un IDE particular para el desarrollo.

	\item Se asume que el dispositivo Holter esta diseñado sobre la arquitectura MSP430 de Texas Instruments.

	\item Se asume la posibilidad de reutilizar código propiedad de ECCOSUR para el desarrollo del software.

	\item Se asume que la memoria flash interna del microcontrolador TI MSP430FR5994 de 2 MB es suficiente para alojar simultáneamente aplicación, bootloader e imagen de restauración.

	\item De ser posible alojar una imagen de restauración, se asume la dependencia del método swap A/B para la carga de la imagen.
\end{itemize}


\subsection{Requisitos futuros}
\label{sec:future_requirements}

\begin{itemize}
	\item Actualización por método FUOTA (depende de una actualización de hardware que incluya un módem LPWAN).
	
	\item Desarrollo de \textit{bootloader updater}, programa que actualiza al bootloader.

	\item Se asume que el dispositivo Holter esta diseñado sobre la arquitectura MSP430 de Texas Instruments.

	\item Se asume la posibilidad de reutilizar código propiedad de ECCOSUR para el desarrollo del software.

	\item Se asume que la memoria flash interna del microcontrolador TI MSP430FR5994 de 2MB es suficiente para alojar simultáneamente aplicación, bootloader e imagen de restauración.

	\item De ser posible alojar una imagen de restauración, se asume la dependencia del método \textit{swap A/B} para la carga de la imagen.
\end{itemize}


\section{Requisitos específicos}
\label{sec:specific_requirements}

\begin{enumerate}
	\item Requerimientos funcionales
		\begin{enumerate}
			\item El proyecto será desarrollado y mantenido mediante control de versiones \textit{GIT}.
			\item El proyecto será desarrollado en lenguaje C, compilado con \textit{TI ARM-CGT 20.2.2.LTS} siguiendo la guía de estilos y analizado 
			estáticamente con la herramienta \textit{Splint}.
			\item El \textit{bootmanager} implementará \textit{branching} de manera segura.
			\item El \textit{bootmanager} implementará \textit{secureboot} para chequeos de origen de firmware antes de un salto a aplicación. 
			\item El \textit{bootloader} actualizará correctamente la versión de firmware actual por una nueva versión certificada 	
			(\textit{Update}).
			\item El \textit{bootloader} restaurará el sistema a la última versión estable instalada en caso de cualquier defecto en la 
			instalación o en el funcionamiento de nuevas versiones \textit{Rollback}.
			\item El \textit{bootloader} soportará comunicación USB para descargar imágenes de firmware.
			\item El \textit{bootloader} soportará el formato de imagen binaria \textit{Motorola SREC}.
			\item El \textit{bootloader} implementará funciones de redundancia cíclica, \textit{CRC} y \textit{checksum} tanto durante la descarga 
			USB como la instalación.
		\end{enumerate}
	\item Requerimientos de documentación
		\begin{enumerate}
			\item El código será documentado con la herramienta \textit{Doxygen}.
			\item El ciclo de vida del presente proyecto será respaldado por una memoria técnica.
			\item La integración del bootloader en otras aplicaciones y casos de uso será facilitada mediante un manual de integración.
		\end{enumerate}
	\item Requerimiento de testing
		\begin{enumerate}
			\item Se desarrollará un script en lenguaje \textit{Python} que cumpla con la funcionalidad mínima de \textit{host} para la descarga USB 
			de la imagen.
			\item Se validará el correcto funcionamiento del bootloader multi-etapa utilizando el script de \textit{Python} como \textit{host}.
		\end{enumerate}
	\item Requerimientos opcionales
		\begin{enumerate}
			\item En caso de que se desee un \textit{bootmanager} y \textit{bootloader}, serán dejados fuera de la sección protegida de la memoria.
		\end{enumerate}
\end{enumerate}


\subsection{Interfaces externas}
\label{sec:external_interfaces}

\begin{itemize}
	\item El bootloader soportará comunicación USB [BTL-REQ-001].
	
	\item El bootloader descargará la imagen binaria a través de una aplicación ``Host'' unívoca [BTL-REQ-002].
	
	\item El bootloader soportorá el formato de imagen binaria \textit{Motorola SREC} [BTL-REQ-003].
\begin{itemize}


\subsection{Funciones}
\label{sec:functions}

\begin{enumerate}
	\item Descarga:
		\begin{enumerate}
			\item La aplicación ``Host'' deberá enviar la imagen en paquetes y realizar control de flujo con el bootloader multi-etapa [BTL-REQ-004].
			
			\item El tamaño de los paquetes de datos (sin metadatos) deberá ser configurable por el bootloader multi-etapa, tomando valores de 256 B, 512 B y 1024 B [BTL-REQ-005].
			
			\item Cada paquete deberá ser verificado mediante redundancia cíclica CRC [BTL-REQ-006].
			
			\item Finalizada la transferencia, ésta deberá ser verificada mediante \textit{Checksum} [BTL-REQ-007].
		\end{enumerate}
		
	\item Actualización y Restauración:
		\begin{itemize}
			\item El bootloader multi-etapa deberá actualizar el firmware actual por la nueva versión introducida por USB sin fallos [BTL-REQ-008].
			
			\item El bootloader multi-etapa deberá restaurar el sistema a la versión anterior estable en caso de fallos en la instalación o defectos en la aplicación [BTL-REQ-009].
			
			\item El método de instalación de la imagen será \textit{swap A/B} [BTL-REQ-010].
		\end{itemize}
		
	\item Integridad:
		\begin{itemize}
			\item La etapa secundaria (\textit{Bootloader} propiamente dicho) actuará como la aplicación en caso de falla, para que el fallo sea seguro. Se considera que existe una falla si la aplicación sufre más de 10 resets en menos de 5 segundos [BTL-REQ-0011].
			
			\item La etapa primaria (\textit{Bootmanager}) deberá detectar fallas contando estos resets almacenándolos en NIR [BTL-REQ-012].
			
			\item En caso de POR, los valores en NIR deberán ser inicializados [BTL-REQ-013].
		\end{itemize}
		
	\item Seguridad:
		\begin{itemize}
			\item La etapa primaria (\textit{Bootmanager}) deberá implementar \textit{Branching} [BTL-REQ-014].
		
			\item La etapa primaria (\textit{Bootmanager}) deberá implementar \textit{Secureboot} [BTL-REQ-015].
			
			\item En caso de que la aplicación no sea válida, el \textit{Bootmanager} saltará a \textit{Bootloader} verificando antes la integridad de este último [BTL-REQ-016].
		\end{itemize}
\begin{enumerate}


\subsection{Requisitos de rendimiento}
\label{sec:performance_requirements}

Se detallarán los requisitos relacionados con la carga que se espera
tenga que soportar el sistema. Por ejemplo, el número de terminales,
el número esperado de usuarios simultaneamente conectados, número de
transacciones por segundo que deberá soportar el sistema, etc.
  También, si es necesario, se especificarpán los requisitos de
datos, es decir, aquellos requisitos que afecten a la información
que se guardará en la base de datos. Por ejemplo, la frecuencia de
uso, las capacidades de acceso y la cantidad de registros que se
espera almacenar (decenas, cientos, miles o millones).


\subsection{Restricciones de diseño}
\label{sec:design_constraints}


Todo aquello que restrinja las decisiones relativas al diseño de la
aplicación: Restricciones de otros estándares, limitaciones del
hardware, etc.


\subsection{Atributos del sistema}
\label{sec:system_attributes}

Se detallarán los atributos de calidad (las "ilities") del
sistema. Fiablidad, manteniblidad, portabilidad, y muy importante,
la seguridad. Deberá especificarse qué tipos de usuarios están
autorizados, o no, a realizar ciertas tareas, y cómo se
implementarán los mecanismos de seguridad (por ejemplo, por medio de
un \emph{login} y una \emph{password}).


\subsection{Otros requisitos}
\label{sec:other_requirements}

Cualquier otro requisito que no encaje en otra sección.

\newpage


\section{Apéndices}
\label{sec:appendix}

Puede contener todo tipo de información relevante para la ERS pero
que, propiamente, no forme parte de la ERS. Por ejemplo:

\begin{enumerate}
\item Formatos de entrada/salida de datos, por pantalla o en listados.

\item Resultados de análisis de costes.

\item Restricciones acerca del lenguaje de programación.
\end{enumerate}
\end{document}
