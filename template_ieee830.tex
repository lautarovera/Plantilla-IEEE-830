% Created 2020-04-24 vie 23:03
% Intended LaTeX compiler: pdflatex
\documentclass[12pt,a4paper]{article}
%------------------------------ PACKAGES -----------------------------%
\usepackage[utf8]{inputenc}
\usepackage[T1]{fontenc}
\usepackage{graphicx}
\usepackage{grffile}
\usepackage{longtable}
\usepackage{wrapfig}
\usepackage{rotating}
\usepackage[normalem]{ulem}
\usepackage{amsmath}
\usepackage{textcomp}
\usepackage{amssymb}
\usepackage{capt-of}
\usepackage{hyperref}
\usepackage[table]{xcolor}
\usepackage[left=2.00cm, right=2.50cm, top=5cm, bottom=2.00cm]{geometry}
\usepackage{fancyhdr}
\usepackage{tabularx}

%------------------------------ DEFINES ------------------------------%
\def\Author{Lautaro Vera}
\def\Email{lautarovera93@gmail.com}
\def\Title{Bootloader Multi-Etapa}

%------------------------------ SETTINGS -----------------------------%
\fancyhead[RO,L]{\thepage}
\fancyhead[LO]{\emph{\uppercase{\leftmark}}}
\fancyfoot{}
\setlength\headheight{15pt} 
\renewcommand{\headrulewidth}{1.0pt}
\renewcommand{\contentsname}{Índice}
\pagestyle{fancy}
\hypersetup{
 colorlinks = true,
 linkcolor = {black},
 linkbordercolor = {white},
 urlcolor = {blue},
 pdfauthor={},
 pdftitle={ERS-IEEE-830},
 pdfkeywords={},
 pdfsubject={},
 pdfcreator={Emacs 26.2 (Org mode 9.1.9)}, 
 pdflang={Spanish}}
 
%----------------------------- DOCUMENT -----------------------------%
\begin{document}
	\begin{titlepage}
		\centering
		{\scshape\Huge \Title \par}
		\vspace{5cm}
		{\bfseries\LARGE Especificación de Requerimientos de Software \par}
		\vspace{1cm}
		{\scshape\Large IEEE Std. 830-1998 \par}
		\vfill
		{\Large \Author \\ (\href{mailto:\Email}{\Email})\par}
		\vfill
		{7 de Noviembre 2021 \par}
	\end{titlepage}

\newgeometry{left=2.00cm, right=2.50cm, top=2.50cm, bottom=2.00cm}

\section*{Registro de cambios}
\label{sec:registro}


\begin{table}[ht]
\label{tab:registro}
	\centering
	\begin{tabularx}{\linewidth}{@{}|c|X|c|c|@{}}
	\hline
	\rowcolor[HTML]{C0C0C0} 
	Versión & \multicolumn{1}{c|}{\cellcolor[HTML]{C0C0C0}Descripción} & Autor       & Fecha      \\ \hline
	A      & Versión original                                          & \Author & 07/10/2021 \\ \hline
	%1      & Se completa hasta el punto 5 inclusive                 & 04/11/2021 \\ \hline
	%2      & Se completa hasta el punto 7 inclusive
	%		  Se puede agregar algo más \newline
	%		  En distintas líneas \newline
	%		  Así                                                    & dd/mm/aaaa \\ \hline
	%3      & Se completa hasta el punto 11 inclusive                & dd/mm/aaaa \\ \hline
	%4      & Se completa el plan	                                 & dd/mm/aaaa \\ \hline
	\end{tabularx}
\end{table}

\pagebreak

\tableofcontents

\newpage

\section{Introducción}
\label{sec:introduction}

En esta sección se proporcionará una introducción a todo el
documento de Especificación de Requisitos Software(ERS). Consta de
varias subsecciones: propósito, ámbito del sistema, definiciones,
referencias y visión general del documento.


\subsection{Propósito}
\label{sec:purpose}

En esta subsección se definirá el propósito del documento ERS y se
especificará a quien va dirigido.


\subsection{Ámbito del sistema}
\label{sec:scope}

En esta subseccion:

\begin{itemize}
\item Se podrá dar un nombre al futuro sistema (p.ej. MiSistema)

\item Se explicará lo que el sistema hará y lo que no hará.

\item Se describirán los beneficios, objetivos y metas que se espera
alcanzar con el futuro sistema.

\item Se referenciarán todos aquellos documentos de nivel superior (p.e en
Industria de sistemas, que incluyen hardware y software, deberían
mantenerse la consistencia con el documento de especificaciones de
requisitos globales del sistema, si existe)
\end{itemize}


\subsection{Definiciones, Acrónimos y Abreviaturas}
\label{sec:glossary}

En esta subsección se definirán todos los términos, acrónimos y
abreviaturas utilizadas en la ERS.


\subsection{Referencias}
\label{sec:references}

En esta subsección se mostrará una lista completa de todos los
documentos referenciados en la ERS


\subsection{Visión general del documento}
\label{sec:overview}

Esta subsección describe brevemente los contenidos y la organización
del resto de la ERS.


\section{Descripción general del documento}
\label{sec:general_description}

En esta sección se describen todos aquellos factores que afectan al
producto y a sus requisitos. No se describen los requisitos, sino su
contexto. Esto permitirá definir con detalle los requisitos en la
sección 3, haciendo que sean más fáciles de entender.

Normalmente, esta sección consta de las siguientes subsecciones:
Perspectiva del producto, funciones del producto, características de
los usuarios, restricciones, factores que se asumen y futuros
requisitos.


\subsection{Perspectiva del producto}
\label{sec:product_perspective}

Esta subsección debe relacionar el futuro sistema (producto
software) con otros productos. Si el producto es totalmente
independiente de otros productos, también debe especificarse
aquí. Si la ERS define un producto que es parte de un sistema mayor,
esta subsección relacionará los requisitos del sistema mayor con la
funcionalidad del producto mayor y el producto aquí descripto. Se
recomienda utilizar diagramas de bloques.


\subsection{Funciones del producto}
\label{sec:product_functions}

En esta subsección de la ERS se mostrará un resumen, a grandes
rasgos, de las funciones del futuro sistema, por ejemplo, en una ERS
para un programa de contabilidad, esta subsección mostrará que el
sistema soportará el mantenimiento de cuentas, mostrará el estado de
las cuentas y facilitará la facturación, sin mencionar el enorme
detalle que cada una de estas funciones requiere.

Las funciones deberán mostrarse de forma organizada, y pueden
utilizarse gráficos, siempre y cuando dichos gráficos reflejen las
relaciones entre funciones y no el diseño del sistema.


\subsection{Características de los usuarios}
\label{sec:user_characteristics}

Esta subsección describirá las características generales de los
usuarios del producto, incluyendo nivel educacional, experiencia y
experiencia técnica.


\subsection{Restricciones}
\label{sec:constraints}

Esta subsección describirá aquellas limitaciones que se imponen
sobre los desarrolladores del producto.

\begin{itemize}
\item Políticas de la empresa.

\item Limitaciones del hardware.

\item Interfaces con otras aplicaciones.

\item Operaciones paralelas.

\item Funciones de auditoría

\item Funciones de control.

\item Lenguaje(s) de programación

\item Protocolos de comunicación.

\item Requisitos de habilidad.

\item Criticalidad de la aplicación.

\item Consideraciones acerca de la seguridad.
\end{itemize}


\subsection{Suposiciones y dependencias}
\label{sec:assumptions_dependencies}

Esta subsección de la ERS describirá aquellos factores que, si
cambian, pueden afectar a los requisitos. Por ejemplo, los
requisitos pueden presuponer una cierta organización de ciertas
unidades de la empresa, o pueden presuponer que  el sistema correrá
sobre cierto sistema operativo. Si cambian dichos detalles en la
organización de la empresa, o si cambian ciertos detalles técnicos,
como el sistema operativo, puede ser necesario revisar y cambiar los
requisitos. 


\subsection{Requisitos futuros}
\label{sec:future_requirements}

Esta subsección esbozará futuras mejoras al sistema, que podrán
anlizarse e implementarse en el futuro.   


\section{Requisitos específicos}
\label{sec:specific_requirements}

Esta sección contiene los requisitos a un nivel de detalle suficiente
como para permitir a los diseñadores diseñar un sistema que
satisfaga estos requisitos, y demuestren si el sistema satisface, o
no, los requisitos. Todo requisito aquí especificado describirá
comportamientos externos del sistema, perceptibles por parte de los
usuarios, operadores y otros sistemas. Esta es la sección más larga
e importante de la ERS. Deberán aplicarse los siguientes principios:

\begin{itemize}
\item El documento debería ser perfectamente legible por personas de muy
distintas formaciones e intereses.

\item Deberán referenciarse aquellos documentos relevantes que poseen
alguna influencia sobre los requisitos.

\item Todo requisito deberá ser unívocamente identificable mediante algún
código o sistema de numeración adecuado.

\item Lo ideal, aunque en la práctica no siempre realizable, es que los
requisitos posean las siguientes características: 

\begin{itemize}
\item \textbf{Corrección:} La ERS es correcta si y sólo si todo requisito que
figura aquí(y que será implementado en el sistema) refleja alguna
necesidad real. La corrección de la ERS implica que el sistema
implementado será el deseado.

\item \textbf{No ambiguos:} Cada requisito tiene una sola interpretación. Para
eliminar la ambigüedad inherente a los requisitos expresados en
lenguaje natural, se deberán utilizar gráficos o notaciones
formales. En el caso de utilizar términos que, habitualmente,
poseen más de una interpretación, se definirán con precisión en
glosario.

\item \textbf{Completos:} Todos los requisitos relevantes han sido incluidos en
la ERS. Conviene incluir todas las posibles respuestas del sistema
a los datos de entrada, tanto validos como no válidos.

\item \textbf{Consistentes:} Los requisitos no pueden ser contradictorios. Un
conjunto de requisitos contradictorios no es implementable.

\item \textbf{Clasificados:} Normalmente, no todos los requisitos son igual de
importantes. Los requisitos pueden clasificarse por importancia
(esenciales, condicionales u opcionales) o por estabilidad (cambios
que se espera que afecten al requisito). Esto sirve, ante todo,
para no emplear excesivos recursos en implementar requisitos no
esenciales.

\item \textbf{Verificables:} La ERS es verificalble si y sólo si todos sus
requisitos son verificables. Un requisito es verificable
(testeable) si existe un proceso finito y no costoso para
demostrar que el sistema cumple con el requisito. Un requisito
ambiguo no es, en general, verificable. Expresiones como a veces,
bien, adecuado, etc introducen ambigüedad en los
requisitos. Requisitos como "en caso de accidente la nube tóxica
no se extenderá más allá de 25km" no es verificable por el alto
costo que conlleva.

\item \textbf{Modificables:} La ERS es modificable si y sólo si se encuentra
estructurada de forma que los cambios a los requisitos puedan
realizarse de forma fácil, completa y consistente. La utilización
de herramientas automáticas de gestión de requisito (por ejemplo
RequisitePro o Doors) facilitan enormemente esta tarea.

\item \textbf{Trazables:} La ERS es trazable si se conoce el origen de cada
requisito y facilita la referencia de cada requisito a los
componentes y de la implementación. La trazabilidad hacia atrás
indica el origen (documento, persona, etc) de cada requisito. La
trazabilidad hacia delante de un requisito R indica qué
componentes del sistema son los que realizan el registro R.
\end{itemize}
\end{itemize}


\subsection{Interfaces externas}
\label{sec:external_interfaces}

Se describirán los requisitos que afecten a la interfaz de usuario,
interfaz con otros sistemas (hardware y software) e interfaces de comunicaciones.


\subsection{Funciones}
\label{sec:functions}

Esta subsección (quizás la más larga del documento) deberá
especificar todas aquellas acciones (funciones) que deberá llevar a
cabo el software. Normalmente (aunque no siempre) son aquellas
acciones expresables como "el sistema deberá \ldots{}" Si se considera
necesario, podrán utilizarse notaciones gráficas y tablas, pero
siempre supeditadas al lenguaje natural, y no al revés.

Es importante tener en cuenta que, en 1983, el estándar de IEEE 830
establecía que las funciones deberían expresarse como una jerarquía
funcional (en paralelo con los DFDs propuestas por el análisis
estructurado). Pero el estándar de IEEE 830, en sus últimas
versiones, ya permite organizar esta subsección de múltiples formas,
y sugiere, entre otras, las siguientes:


\begin{itemize}
\item Por tipos de usuarios: 
    Distintos usuarios poseen distintos requisitos. Para cada clase de
usuario que exista en la organización, se especificarán los
requisitos funcionales que le afecten o tengan mayor relación con
sus tareas.
\end{itemize}


\begin{itemize}
\item Por objetos:
   Los objetos son identidades del mundo real que serán reflejadas en
el sistema. Para cada objeto, se detallarán sus atributos y sus
funciones. Los objetos pueden agruparse en clases. Esta organización
de la ERS no quiere decir que el diseño del sistema siga el
paradigma de Orientación a Objetos.
\end{itemize}


\begin{itemize}
\item Por estímulos: 
  Se especificarán los posibles estímulos que recibe el sistema y las
funciones relacionadas con dicho estímulo.
\end{itemize}


\begin{itemize}
\item Por jerarquía funcional: 
   Si ninguna de las anteriores alternativas resulta de ayuda, la
funcionalidad del sistema se especificará como una jerarquía de
funciones que comparten entradas, salidas o datos internos. Se
detallarán las funciones (entrada, proceso, salida) y las
subfunciones del sistema. Esto no implica que el diseño del sistema
deba realizarse según el paradigma de diseño estructurado.
\end{itemize}


Para organizar esta subsección de la ERS se elegirá alguna de las
anteriores alternativas, o incluso alguna otra que se considere más
conveniente. Deberá, eso sí, justificarse el porqué de tal elección.



\subsection{Requisitos de rendimiento}
\label{sec:performance_requirements}

Se detallarán los requisitos relacionados con la carga que se espera
tenga que soportar el sistema. Por ejemplo, el número de terminales,
el número esperado de usuarios simultaneamente conectados, número de
transacciones por segundo que deberá soportar el sistema, etc.
  También, si es necesario, se especificarpán los requisitos de
datos, es decir, aquellos requisitos que afecten a la información
que se guardará en la base de datos. Por ejemplo, la frecuencia de
uso, las capacidades de acceso y la cantidad de registros que se
espera almacenar (decenas, cientos, miles o millones).


\subsection{Restricciones de diseño}
\label{sec:design_constraints}


Todo aquello que restrinja las decisiones relativas al diseño de la
aplicación: Restricciones de otros estándares, limitaciones del
hardware, etc.


\subsection{Atributos del sistema}
\label{sec:system_attributes}

Se detallarán los atributos de calidad (las "ilities") del
sistema. Fiablidad, manteniblidad, portabilidad, y muy importante,
la seguridad. Deberá especificarse qué tipos de usuarios están
autorizados, o no, a realizar ciertas tareas, y cómo se
implementarán los mecanismos de seguridad (por ejemplo, por medio de
un \emph{login} y una \emph{password}).


\subsection{Otros requisitos}
\label{sec:other_requirements}

Cualquier otro requisito que no encaje en otra sección.

\newpage


\section{Apéndices}
\label{sec:appendix}

Puede contener todo tipo de información relevante para la ERS pero
que, propiamente, no forme parte de la ERS. Por ejemplo:

\begin{enumerate}
\item Formatos de entrada/salida de datos, por pantalla o en listados.

\item Resultados de análisis de costes.

\item Restricciones acerca del lenguaje de programación.
\end{enumerate}
\end{document}
